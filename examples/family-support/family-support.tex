\documentclass{article}
% \usepackage{a4}
\usepackage{graphicx}
% \usepackage{amsfonts}
% \usepackage{amsmath} 
% \usepackage{amssymb}
\usepackage{alltt}

%\newcommand{\sect}[2]{\smallskip\noindent{\bf\S#1.} #2 \smallskip}

\newenvironment{sect}[1]{\smallskip\noindent{\bf\S#1.}}{\smallskip}

% \usepackage[pdftex, colorlinks=true, pdfstartview=FitH, 
% linkcolor=blue, citecolor=blue, urlcolor=blue, plainpages=false, 
% pdfpagelabels, pdftitle={A Formal Description of the Rulewise
% Formalism and Method}, pdfauthor={ESTRELLA Project}]{hyperref}

\sloppy

\title{Formal Models of Legislation \\
Family Law Exercise}
\author{XML Summer School}

% \correspondingauthor{}
% \runningauthor{}
% \http{http://www.fokus.fraunhofer.de}

% \ESTRELLAreport{WP4 Whitepaper}
% \ESTRELLAreportdate{April 2007}

% \status{Version 0.1}{}

% \acknowledge{}

\begin{document}

\maketitle

Here we present a small, toy legal domain, roughly based on German family  law, for use in learning the Legal Knowledge Representation Format (LKIF).  The question addressed is whether or not a descendent of some person, typically a child or grandchild, is obligated to pay financial support to the ancestor.

\begin{sect}{1601 BGB (Support Obligations)}
Relatives in direct lineage are obligated to support each other.
\end{sect}

\begin{sect}{1589 BGB (Direct Lineage)}  
A relative is in direct lineage if he is a descendent or ancestor. For example, parents, grandparents and great grandparents are in direct lineage.
\end{sect}

\begin{sect}{1741 BGB (Adoption)} 
For the purpose of determining support obligations, an adopted child is a descendent of the adopting parents.
\end{sect}

\begin{sect}{1590 BGB (Relatives by Marriage)}
There is no obligation to support the relatives of a spouse (husband or wife), such as a mother-in-law or father-in-law.
\end{sect}

\begin{sect}{1602 BGB (Neediness)}
Only needy persons are entitled to support by family members.  A person is needy only if unable to support himself.
\end{sect}

\begin{sect}{1603 BGB (Capacity to Provide Support)}
A person is not obligated to support relatives if he does not have the capacity to support others, taking into consideration his income and assets as well as his own reasonable living expenses.
\end{sect} 

\begin{sect}{1611a BGB (Neediness Caused By Immoral Behavior)}
A needy person is not entitled to support from family members if his neediness was caused by his own immoral behavior, such as gambling, alcoholism, drug abuse or an aversion to work.
\end{sect}

\begin{sect}{91 BSHG (Undue Hardship)}
A person is not entitled to support relatives if this would cause him undue hardship.
\end{sect}


\bibliographystyle{abbrv}
\bibliography{/Users/tgo/Documents/Bibliography/tfgordon}
\end{document}




